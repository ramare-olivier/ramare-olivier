\input amstex
\magnification=\magstep1
\documentstyle {amsppt}
\NoBlackBoxes
\refstyle{B}

%%% Version du 28 Mars 2000

\def\goes{\mathrel{\rightarrow}}
\def\Log{\operatorname{Log}}
\def\section#1{\goodbreak\smallskip\noindent{\bf #1}}
\def\fin{$\diamond\diamond\diamond$\enddemo}
\def\Ker{\operatorname{Ker}}
\def\1{1\!\!\!1}

\document

\topmatter
\title
Le th\'eor\`eme de Mertens sur la distribution des nombres premiers en
progressions arithm\'etiques et la non-annulation des fonctions L en 1
\endtitle
\rightheadtext{Le th\'eor\`eme de Mertens}
\author
Olivier Ramar\'e
\endauthor
\abstract
Une preuve qui ne demande que peu de pr\'erequis et d\'emontre de
belles choses~!

Version du 28 Mars 2000.
\endabstract
\endtopmatter

\section{I. R\'esultats.}

Soit $q$ un entier $\ge 1$ et $a$ un entier premier \`a $q$. Nous
d\'emontrons le th\'eor\`eme de Mertens qui affirme que
$$
\sum\Sb d\le D\\d\equiv a[q]\endSb
\frac{\Lambda(d)}{d}=\frac1{\phi(q)}\Log D+\Cal O(1),
\leqno(1.1)
$$
o\`u le symbole $\Cal O$ d\'epend de $q$. Nous n'utiliserons que les
propri\'et\'es \'el\'ementaires des caract\`eres de Dirichlet,
une borne sup\'erieure de type Tchebyschef (le lemme~1 ci-dessous)
et les identit\'es de convolution $\Lambda\star\1=\Log$ et
$\mu\star\Log=\Lambda$.

Nous en d\'eduirons qu'il existe trois constantes strictement
positives $C_1(q)$, $C_2(q)$ et $C_3(q)$ telles que
$$
C_1(q) \,\frac{D}{\phi(q)}\le
\sum\Sb d\le D\\d\equiv a[q]\endSb\Lambda(d)
\le C_2(q) \,\frac{D}{\phi(q)}\qquad,\quad(D\ge C_3(q)).
$$

\section{II. Pr\'eliminaires.}

\proclaim{Lemme~1 (Tchebyschef)}
Il existe une constante strictement positive $C$ telle que
$$
\sum_{d\le D}\Lambda(d)\le C\, D\qquad(D\ge1)
$$
\endproclaim

\proclaim{Lemme~2}
Nous avons, pour tout caract\`ere $\chi$ non principal modulo $q$ ~:
$$
\left\{\aligned
&\sum_{d\le D}\frac{\Lambda(d)}{d}=\Log X+\Cal O(1)
\\&
\sum_{n\le N}\frac{\chi(n)}{n}= L(1,\chi)+\Cal O(1/N)
\\&
\sum_{n\le N}\frac{\chi(n)\Log n}{n}= -L'(1,\chi)+\Cal O((\Log N)/N)
\\&
\sum_{n\le N}\frac{\chi(n)}{\sqrt{n}}= L(1/2,\chi)+\Cal O(1/N^{1/2})
\endaligned\right.
$$
\endproclaim
Il faut remarquer que dans les \'egalit\'es ci-dessus, ce qui nous
int\'eresse c'est que $L(1,\chi)$, $L'(1,\chi)$ et $L(1/2,\chi)$
soient des constantes et non leur forme particuli\`ere.

\demo{Preuve}
Nous partons de
$$
\sum_{n\le N}\Log n=N\Log N+\Cal O(N).
$$
Comme $\Log=\1\star\Lambda$, nous en d\'eduisons
$$
\aligned
N\Log N+\Cal O(N)
&=\sum\Sb d,m\ge 1\\ dm\le N\endSb\Lambda(d)
=
\sum_{ d\le N}\Lambda(d)\left(
\frac{N}{d}+\Cal O(1)
\right)
\\&=N\sum_{d\le N}\frac{\Lambda(d)}{d}+\Cal O(N)
\endaligned
$$
grace au lemme~1, ce qui d\'emontre la premi\`ere \'egalit\'e.
Les suivantes suivent toutes le m\^emes sch\'ema et nous ne montrons
que la premi\`ere. Nous avons
$$
\sum_{n\le N}\frac{\chi(n)}{n}
=L(1,\chi)-\sum_{n>N}\frac{\chi(n)}{n}
=L(1,\chi)-\int_{N}^{\infty}\frac{\sum_{N<n\le t}\chi(n)}{t^2}dt
$$
en utilisant
$\frac1n=\int_n^{\infty}dt/t^2$,
et le r\'esultat provient ensuite de ce que $|\sum_{n\le t}\chi(n)|\le q$.
\fin

\section{III. Deux lemmes.}

\proclaim{Lemme~3}
Soit $\chi$ un caract\`ere modulo $q$ non principal.
Nous avons
$$
\left[L(1,\chi)\neq0\quad\implies\quad
\sum_{d\le D}\frac{\chi(d)\Lambda(d)}{d}=\Cal O(1)\right].
$$
\endproclaim

\demo{Preuve}
Puisque $\Lambda\star\1=\Log$, nous avons
$$
\aligned
\sum_{n\le N}\frac{\chi(n)\Log n}{n}
&=
\sum_{d\le N}\frac{\chi(d)\Lambda(d)}{d}\sum_{m\le
N/d}\frac{\chi(m)}{m}
\\&=
\sum_{d\le N}\frac{\chi(d)\Lambda(d)}{d}
\left\{
L(1,\chi)+\Cal O\bigg(\frac{d}{N}\bigg)
\right\}
\\&=
L(1,\chi)\sum_{d\le N}\frac{\chi(d)\Lambda(d)}{d}
+\Cal O\bigg(\sum_{d\le N}\Lambda(d)/N\bigg).
\endaligned
$$
Les lemmes~1 et~2 nous permettent de conclure
facilement.
\fin

\proclaim{Lemme~4}
Soit $\chi$ un caract\`ere modulo $q$ non principal.
Nous avons
$$
\left[L(1,\chi)=0\quad\implies\quad
\sum_{d\le D}\frac{\chi(d)\Lambda(d)}{d}=-\Log D+\Cal O(1)\right].
$$
\endproclaim

\demo{Preuve}
Nous avons
$$
\sum_{n|d}\mu(n)\Log\frac{D}{n}
=
\sum_{n|d}\mu(n)\Log\frac{D}{d}
+\sum_{n|d}\mu(n)\Log\frac{d}{n}
=\cases
\Log D\quad&\text{si}\quad d=1\\
\Lambda(d)\quad&\text{si}\quad d>1.
\endcases
$$
Il vient alors
$$
\aligned
\sum_{d\le D}\frac{\chi(d)\Lambda(d)}{d}
&=-\Log D
+\sum_{d\le D}\frac{\chi(d)}d\sum_{n|d}\mu(n)\Log\frac{D}{n}
\\&=
-\Log D
+\sum_{n\le D}\frac{\mu(n)\chi(n)}{n}\Log\frac{D}{n}
\sum_{m\le D/n}\frac{\chi(m)}{m}
\\&=
-\Log D
+\sum_{n\le D}\frac{\mu(n)\chi(n)}\Log\frac{D}{n}
\left\{L(1,\chi)+\Cal O\bigg(\frac{n}{D}\bigg)\right\}.
\endaligned
$$
Puisque nous supposons que $L(1,\chi)=0$, et moyennant de rappeler que
$$
\sum_{n\le D}\Log\frac{D}{n}=\Cal O(D),
$$
nous obtenons bien le r\'esultat annonc\'e.
\fin

\section{IV. Premi\`eres d\'eductions.}

Nous avons
$$
\frac1{\phi(q)}\sum_{\chi\mod q}\overline{\chi(a)}\chi(d)
=\cases
1\quad&\text{si}\quad d\equiv a[q]\\
0\quad&\text{si}\quad d\not\equiv a[q]\\
\endcases
$$
ce qui nous donne
$$
\sum\Sb d\equiv a[q]\\ d\le D\endSb\frac{\Lambda(d)}d=
\frac1{\phi(q)}\sum_{\chi\mod q}\overline{\chi(a)}
\sum\Sb d\le D\endSb\frac{\chi(d)\Lambda(d)}d.
$$
D\'efinissons $\delta(\chi)$ par
$$
\delta(\chi)=\cases
1\quad&\text{si}\quad \chi=\chi_0\\
-1\quad&\text{si}\quad \chi\neq\chi_0\quad\text{et}\quad L(1,\chi)=0\\
0\quad&\text{si}\quad \chi\neq\chi_0\quad\text{et}\quad L(1,\chi)\neq0\\
\endcases
$$
(o\`u $\chi_0$ est le caract\`ere principal) de telle sorte que
$$
\sum\Sb d\le D\endSb\frac{\chi(d)\Lambda(d)}d=\delta(\chi)\Log D+\Cal O(1).
$$
Nous obtenons alors
$$
\sum\Sb d\equiv a[q]\\ d\le D\endSb\frac{\Lambda(d)}d=
\frac{1}{\phi(q)}\left(
\sum_{\chi\mod q}\overline{\chi(a)}\delta(\chi)
\right)\Log D+\Cal O(1).
\leqno(4.1)
$$
En sp\'ecialisant en $a=1$, et en remarquant que le membre de gauche
de l'\'egalit\'e pr\'ec\'edente est positif ou nul, nous obtenons
$$
0\le\sum_{\chi\mod q}\delta(\chi)
=1-\sum\Sb \chi/ L(1,\chi)=0\endSb 1
$$
tant et si bien qu'il y a au plus un caract\`ere modulo $q$ pour
lequel $L(1,\chi)=0$. Comme
$L(1,\overline{\chi})=\overline{L(1,\chi)}$, ce caract\`ere est
n\'ecessairement r\'eel.

\section{V. La non-annulation de $L(1,\chi)$ pour $\chi$ r\'eel.}

Pour montrer que $L(1,\chi)\neq0$ lorsque $\chi$ est r\'eel,
remarquons tout d'abord que
$$
\sum_{d|n}\chi(d)\ge0
$$
et m\^eme $\ge 1$ si $n$ est un carr\'e. En effet, comme la fonction
$\1\star\chi$ est multiplicative, il nous suffit de v\'erifier ces
propri\'et\'es sur les puissances de nombres premiers. Si $\chi(p)=1$,
alors $\1\star\chi(p^{\nu})=\nu+1$ et si $\chi(p)=-1$, alors
$\1\star\chi(p^{\nu})=((-1)^{\nu}+1)/2$, ce qui d\'emontre bien ce que
nous avons annonc\'e.

Donnons-nous deux param\`etres $M,D\ge1$ et tels que $MD=N$.
Il vient
$$
\aligned
\sum_{n\le N}\frac{{\sum_{d|n}\chi(d)}}{\sqrt{n}}
&=
\sum_{d, m/ dm\le N}\frac{\chi(d)}{\sqrt{d}}\frac1{\sqrt{m}}
\\&=
\sum_{d\le D}\frac{\chi(d)}{\sqrt{d}}\sum_{m\le N/d}\frac1{\sqrt{m}}
+
\sum_{m\le M}\frac1{\sqrt{m}}\sum_{D<d\le N/m}\frac{\chi(d)}{\sqrt{d}}
\\&=
\sum_{d\le D}\frac{\chi(d)}{\sqrt{d}}
\left\{
\frac12\sqrt{\frac{N}{d}}+c+\Cal O\bigg(\sqrt{\frac{d}{N}}\bigg)
\right\}
+\sum_{m\le M}\frac1{\sqrt{m}}\Cal O(1/\sqrt{D})
\endaligned
$$
(car $\sum_{m\le M}m^{-1/2}=M^{1/2}/2+c+\Cal O(M^{-1/2})$),
soit finalement
$$
\sum_{n\le N}\frac{{\sum_{d|n}\chi(d)}}{\sqrt{n}}
=\frac12\sqrt{{N}}(L(1,\chi)+\Cal O(1/D))
+\Cal O(1+ D N^{-1/2}+M^{1/2}D^{-1/2}).
$$
Nous prenons alors $D=M=N^{1/2}$, ce qui nous donne
$$
\sum_{n\le N}\frac{{\sum_{d|n}\chi(d)}}{\sqrt{n}}
=\frac12\sqrt{{N}}L(1,\chi)
+\Cal O(1)
.
$$
Par ailleurs
$$
\sum_{n\le N}\frac{{\sum_{d|n}\chi(d)}}{\sqrt{n}}
\ge \sum_{\ell^2\le N}\frac{1}{\sqrt{\ell^2}}
\ge \frac12\Log N+\Cal O(1).
$$
Par cons\'equent
$$
L(1,\chi)\ge\frac{\Log N+\Cal O(1)}{\sqrt{N}}
$$
et il suffit de prendre $N$ assez grand pour obtenir $L(1,\chi)\ne0$.

\section{VI. Conclusion.}

L'\'equation $(4.1)$ alli\'e au fait que $\delta_\chi=0$ pour tout
caract\`ere non principal nous donne le th\'eor\`eme de Mertens
$(1.1)$.

Pour en d\'eduire des in\'egalit\'es de type Tchebyschef $(1.2)$, nous
proc\'edons comme suit. Donnons-nous un param\`etre $\alpha\in]0,1[$.
Nous avons
$$
\sum\Sb \alpha D< d\le D\\ d\equiv a[q]\endSb\frac{\alpha D\Lambda(d)}{d}
\le
\sum\Sb \alpha D< d\le D\\ d\equiv a[q]\endSb\Lambda(d)
\le
\sum\Sb \alpha D< d\le D\\ d\equiv a[q]\endSb\frac{D\Lambda(d)}{d}
$$
o\`u il nous suffit d'appliquer $(1.1)$ pour obtenir
$$
\alpha D\left(\frac{\Log(1/\alpha)}{\phi(q)}+\Cal O(1)\right)
\le
\sum\Sb \alpha D< d\le D\\ d\equiv a[q]\endSb\Lambda(d)
\le
D\left(\frac{\Log(1/\alpha)}{\phi(q)}+\Cal O(1)\right)
$$
et nous prenons alors $\alpha$ suffisamment petit de sorte que les
quantit\'es $\frac{\Log(1/\alpha)}{\phi(q)}+\Cal O(1)$
ci-dessus soient toutes les deux $\ge1$. Nous ajoutons les
contributions de $D$ \`a $\alpha D$, puis de $\alpha D$ \`a $\alpha^2
D$, etc, ce qui prouve $(1.2)$.


\section{VII. Commentaires.}

Les lignes qui suivent sont emprunt\'ees \`a plusieurs preuves
plus ou moins classiques dont je ne saurai reconna\^\i tre les
auteurs. Il faut remarquer que $(4.1)$ parle d\'ej\`a de la
c\'el\`ebre constante~2 du th\'eor\`eme de Brun-Tichmarsh~: si il
existe un z\'ero exceptionnel, disons ici $L(1,\chi)=0$ dans notre
formalisme simplifi\'e, alors certaines classes contiennent deux fois
plus de nombres premiers que la valeur esp\'er\'ee. Plus remarquable
encore, ces classes forment un sous-groupe...

L'argument de la partie $V$ si il a l'avantage de s'int\'egrer
agr\'eablement au reste du d\'eveloppement donne une pi\`etre
minoration de $L(1,\chi)$ en fonction de $q$ d\`es que l'on explicite
la d\'ependance dans ce param\`etre.


\enddocument

%%% Local Variables: 
%%% mode: plain-tex
%%% TeX-master: t
%%% End: 
