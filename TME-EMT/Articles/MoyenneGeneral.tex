\input amstex
\magnification=\magstep1
\documentstyle {amsppt}
\NoBlackBoxes
\refstyle{B}

%%% Version du 20 Mars 2000

\def\goes{\mathrel{\rightarrow}}
\def\Log{\operatorname{Log}}
\def\section#1{\goodbreak\smallskip\noindent{\bf #1}}
\def\fin{$\diamond\diamond\diamond$\enddemo}
\def\Ker{\operatorname{Ker}}

\document

\topmatter
\title
Moyennes de fonctions multiplicatives positives
\endtitle
%\rightheadtext{}
\author
Olivier Ramar\'e
\endauthor
\abstract
Nous pr\'esentons ici deux th\'eor\`emes g\'en\'eraux qui permettent
d'\'evaluer $G(D)=\sum_{d\le D}g(d)$ o\`u $g$ est une fonction
multiplicative assez g\'en\'erale dont on conna\^\i t les valeurs sur
les puissances de nombres premiers.
La technique de base est remarquablement similaire dans les deux cas.

Ces r\'esultats s'appliquent lorsque peut de compensations
ont lieu dans la somme d\'efinissant $G(D)$ et nous avons pris
l'hypoth\`ese simplificatrice $g\ge0$ qui n'est toutefois pas
essentielle pour obtenir une version un peu plus faible du
th\'eor\`eme~2.

Version pr\'eliminaire du 20 Mars 2000.
\endabstract
\endtopmatter

\section{I. Une premi\`ere borne sup\'erieure.}

Ce premier th\'eor\`eme est efficace lorsque l'ordre moyen de $g$ sur
les puissances de nombres premiers est proche de constant.
Il existe de nombreuses versions de ce r\'esultat, dont la plus
pr\'ecise est due \`a Halberstam \&~Richert (voir l'article de 1979
cit\'e ci-apr\`es). Celle que nous 
pr\'esentons est une l\'eg\`ere modification de ce qui est propos\'e
dans le livre de Tenenbaum r\'ef\'erenc\'e \`a la fin de ce papier.

\proclaim{Th\'eor\`eme~1}
Supposons que $g$ soit positive ou nulle et que
$$
\sum\Sb p\ge 2,\nu\ge1 \\ p^{\nu}\le Q\endSb
g\big(p^{\nu}\big)\Log\big(p^{\nu}\big)
\le KQ\qquad(Q\ge1)
$$
pour une certaine constante $K\ge0$. Alors
$$
\sum_{d\le D}g(d)\le (K+1)\frac{D}{\Log D}\sum_{d\le D}g(d)/d
\qquad(D\ge2).
$$
\endproclaim
\demo{Preuve}
Posons $\tilde G(D)=\sum_{d\le D} g(d)/d$. Alors,
en utilisant $\Log\frac Dd\le\frac Dd$, nous obtenons
$$
\aligned
G(D)\Log D
&=\sum_{d\le D}g(d)\Log\frac{D}{d}+
\sum_{d\le D}g(d)\Log d
\\&\le
D \sum_{d\le D}\frac{g(d)}{d}
+\sum\Sb p\ge 2,\nu\ge1 \\ p^{\nu}\le D\endSb
g\big(p^{\nu}\big)\Log\big(p^{\nu}\big)
\sum\Sb \ell\le D/p^{\nu}\\ (\ell,p)=1\endSb g(\ell)
\endaligned
$$
o\`u l'on obtient le second sommant en \'ecrivant
$$
\Log d=\sum_{p^{\nu}\|d}\Log\big(p^{\nu}\big).
$$
Finalement
$$
\aligned
\sum\Sb p\ge 2,\nu\ge1 \\ p^{\nu}\le D\endSb
g\big(p^{\nu}\big)\Log\big(p^{\nu}\big)
\sum\Sb \ell\le D/p^{\nu}\\ (\ell,p)=1\endSb g(\ell)
&=
\sum\Sb \ell\le D\endSb g(\ell)
\sum\Sb p\ge 2,\nu\ge1 \\ p^{\nu}\le D/\ell\\ (p,\ell)=1\endSb
g\big(p^{\nu}\big)\Log\big(p^{\nu}\big)
\\&\le
\sum\Sb \ell\le D\endSb g(\ell)K\frac{D}\ell
\endaligned
$$
ce qui nous permet de conclure facilement.
\fin

\section{II. Une formule asymptotique.}

Notre second th\'eor\`eme s'applique lui aux fonctions
multiplicatives $g$ telles que $g(p)\simeq\kappa/p$ pour un certain
$\kappa>0$.

Ce second th\'eor\`eme demande des hypoth\`eses bien plus fortes
mais en \'echange nous prouvons une formule asymptotique.
Sa preuve est \`a la base assez simple, mais se complique
du fait que (i) nous prenons en compte la d\'ependance en certains
param\`etres des termes d'erreur (ii) nous tenons compte des valeurs
de $g$ sur les puissances de nombres premiers (iii) nous offrons  un
r\'esultat compl\`etement explicite. L'essentiel de ce th\'eor\`eme se
trouve dans l'article de Halberstam \&~ Richert de 1971 cit\'e ci-dessous, et
une pr\'esentation l\'eg\`erement simplifi\'ee dans le livre de ces
deux m\^emes auteurs.

\proclaim{Th\'eor\`eme~2}
Donnons-nous une fonction multiplicative $g$
et trois param\`etres
r\'eels strictement positifs $\kappa$, $L$ et $A$ tels que
$$
\left\{
\aligned
&\sum\Sb p\ge2, \nu\ge1\\ w<p^{\nu}\le Q\endSb
g\big(p^{\nu}\big)\Log\big(p^{\nu}\big)=
\kappa\Log\frac{Q}{w}+\Cal O^*(L)\qquad(Q>w\ge1),
\\&
\sum_{p\ge2}
\sum_{\nu,k\ge1}g\big(p^k\big)g\big(p^{\nu}\big)\Log\big(p^{\nu}\big)
\le A.
\endaligned
\right.
$$
Alors
$$
\sum_{d\le D}g(d)= C\left(\Log D\right)^{\kappa}
\left(1+\Cal O^*(B/\Log D)\right)\qquad (D\ge\exp(2(L+A)))
$$
avec
$$
\left\{
\aligned
C&=\frac{1}{\Gamma(\kappa+1)}
\prod_{p\ge2}\bigg\{
\bigg(\sum_{\nu\ge0}g\big(p^{\nu}\big)\bigg)
\bigg(1-\frac1p\bigg)^{\kappa}\bigg\},
\\
B&=2(L+A)\big(1+2(\kappa+1)e^{\kappa+1}\big)
\endaligned
\right.
$$
\endproclaim
Notons que dans la plupart des applications, si la d\'ependance en
$L$ peut s'av\'erer importante, celle en $A$ est presque toujours
sans int\'er\^et.

\demo{Preuve}
Le d\'epart est similaire au pr\'ec\'edent~:
$$
\aligned
G(D)\Log D
&=\sum_{d\le D}g(d)\Log\frac{D}{d}+
\sum_{d\le D}g(d)\Log d
\\&=\sum_{d\le D}g(d)\Log\frac{D}{d}+
\sum\Sb p\ge 2,\nu\ge1 \\ p^{\nu}\le D\endSb
g\big(p^{\nu}\big)\Log\big(p^{\nu}\big)
\sum\Sb \ell\le D/p^{\nu}\\ (\ell,p)=1\endSb g(\ell)
\endaligned
$$
et l'on pose
$$
\left\{
\aligned
G_p(X)&=\sum\Sb \ell\le X\\ (\ell,p)=1\endSb g(\ell)
\\T(D)&=\sum_{d\le D}g(d)\Log\frac{D}{d}
=\int_1^D G(t)\frac{dt}{t},
\endaligned\right.
$$
ce qui nous donne
$$
G(D)\Log(D)=T(D)+\sum\Sb p\ge 2,\nu\ge1 \\ p^{\nu}\le D\endSb
g\big(p^{\nu}\big)\Log\big(p^{\nu}\big)G_p(D/p^{\nu}).
$$
De plus
$$
G_p(X)=G_p(X)-\sum_{k\ge1}G_p(X/p^k)
$$
ce qui, joint \`a nos hypoth\`ese, nous donne
$$
\aligned
G(D)\Log(D)
&=T(D)+\sum\Sb p\ge 2,\nu\ge1 \\ p^{\nu}\le D\endSb
g\big(p^{\nu}\big)\Log\big(p^{\nu}\big)G(D/p^{\nu})
+\Cal O^*(A G(D))
\\&=
T(D)+\sum_{d\le D}g(d)\sum\Sb p\ge 2,\nu\ge1 \\ p^{\nu}\le D/d\endSb
g\big(p^{\nu}\big)\Log\big(p^{\nu}\big)
+\Cal O^*(A G(D))
\\&=T(D)(\kappa+1)+\Cal O^*((L+A) G(D))
\endaligned
$$
ce que nous r\'e\'ecrivons en
$$
\aligned
& (\kappa+1)T(D)=G(D)\Log D\ (1+r(D))
\\& r(D)=\Cal O^*\bigg(\frac{L+A}{\Log D}\bigg)\quad D_0=\exp(2(L+A))
\endaligned
$$
que nous regardons comme une \'equation diff\'erentielle. Posons
$$
\exp E(D)
=\frac{(\kappa+1)T(D)}{(\Log D)^{\kappa+1}}
=\frac{G(D)}{(\Log D)^\kappa}(1+r(D))
$$
ce qui nous donne
$$
E'(D)=
\frac{T'(D)}{T(D)}-\frac{(\kappa+1)}{D\Log D}
=\frac{r(D)(\kappa+1)}{(1-r(D))D\Log D}
=\Cal O^*\bigg(\frac{2(L+A)}{D\Log D}\bigg)
\quad(D\ge D_0)
$$
puisque $|r(D)|\le\tfrac12$ si $D\ge D_0$.
Il vient
$$
E(\infty)-E(D)=\int_D^\infty E'(t)dt
=\Cal O^*\bigg(\frac{2(L+A)}{\Log D}\bigg)
\quad(D\ge D_0).
$$
Bref
$$
\frac{G(D)}{(\Log D)^\kappa}=
\frac{1}{1+r(D)}\exp E(D)
=\frac{e^{E(\infty)}}{1+r(D)}\bigg(1+\Cal O^*\bigg(
\frac{2(L+A)}{\Log D}(\kappa+1)e^{\kappa+1}\bigg)\bigg).
$$
Or $1/(1+x)\le 1+2x$ si $0\le x\le\tfrac12$  d'o\`u
$$
\frac{G(D)}{(\Log D)^\kappa}=e^{E(\infty)}
\bigg(1+\Cal O^*\bigg(
\frac{2(L+A)}{\Log D}\big(1+2(\kappa+1)e^{\kappa+1}\big)\bigg)\bigg)
\qquad(D\ge D_0)
$$
ce qui constitue l'essentiel de la d\'emonstration.
Il nous faut \`a pr\'esent expliciter $e^{E(\infty)}=C$.
Remarquons tout d'abord que la preuve ci-dessus est a priori fausse
car $T'(D)\neq G(D)/D$ aux points de discontinuit\'es de $G$, mais
il nous suffit de travailler avec $D$ non entier et de proc\'eder par
continuit\'e.

\bigskip
\noindent{\sl Expression de $C$:}

Pour ce qui est du calcul de la constante, nous avons pour $s$ r\'eel positif
$$
\aligned
D(g,s)
&=\sum_{d\ge1}\frac{g(d)}{d^s}=s\int_1^\infty G(D)\frac{dD}{D^{s+1}}
\\&= s C\int_1^\infty(\Log D)^{\kappa}\frac{dD}{D^{s+1}}
+\Cal O\bigg(
sC\int_1^\infty(\Log D)^{\kappa-1}\frac{dD}{D^{s+1}}
\bigg)
\\&=
C\big(s^{-\kappa}\Gamma(\kappa+1)+\Cal O(s^{1-\kappa}\Gamma(\kappa))\big)
\endaligned
$$
et par cons\'equent
$$
C=\lim_{s\goes0^+} D(g,s)s^{\kappa}\Gamma(\kappa+1)^{-1}
=\lim_{s\goes0^+} D(g,s)\zeta(s+1)^{-\kappa}\Gamma(\kappa+1)^{-1}.
$$
Il est alors assez facile de montrer que le produit
$$
\prod_{p\ge2}\bigg\{
\bigg(\sum_{\nu\ge0}g\big(p^{\nu}\big)\bigg)
\bigg(1-\frac1p\bigg)^{\kappa}\bigg\}
$$
est convergent et \'egale donc $C\Gamma(\kappa+1)$ comme voulu.

% \bigskip
% \noindent{\sl Extension du domaine de validit\'e.}

% Jusqu'\`a pr\'esent, nous n'avons pas utilis\'e la positivit\'e de
% $g$, et nous pourrions aussi probablement
% \'etendre le domaine de validit\'e sans cette hypoth\`ese.

% Comme
% $$
% B(D)/\Log D\ge1\qquad (D\le D_0)
% $$
% il nous suffit de montrer que
% $$
% G(D)\le C(\Log D)^{\kappa}
% \big(1+B(D)/\Log D\big)\qquad(D<D_0).
% $$
% Or
% $$
% \aligned
% G(D)
% &\le
% \prod_{p\le D}
% \bigg(\sum\Sb \nu\ge 0\\ p^{\nu}\le D\endSb g\big(p^{\nu}\big)\bigg)
% \\&\le
% \prod_{p\le D}\bigg\{
% \bigg(\sum\Sb \nu\ge 0\\ p^{\nu}\le D\endSb g\big(p^{\nu}\big)\bigg)
% \bigg(1-\frac1p\bigg)^{\kappa}\bigg\}
% \prod_{p\le D}\bigg(1-\frac1p\bigg)^{-\kappa}
% \endaligned
% $$
% et il s'agit en fait de mesurer la proximit\'e de
% $$
% C_D=\frac1{\Gamma(\kappa+1)}\prod_{p\le D}\bigg\{
% \bigg(\sum\Sb \nu\ge 0\\ p^{\nu}\le D\endSb g\big(p^{\nu}\big)\bigg)
% \bigg(1-\frac1p\bigg)^{\kappa}\bigg\}
% $$
% de $C$. Soit donc $D'\ge D$. Consid\'erons
% $$
% \aligned
% \Log(C_{D'}/C_D)
% =
% &\sum_{p\le D}
% \Log\left(
% 1+\frac{\sum_{D<p^{\nu}\le D'}g(p^{\nu})}{1+\sum_{p^{\nu}\le D}g(p^{\nu})}
% \right)
% \\&+
% \sum_{D<p\le D'}
% \Log\left(
% 1+\sum_{p^{\nu}\le D'}g(p^{\nu})
% \right)
% +\kappa\sum_{D<p\le D'}\Log\big(1-\frac1p\big).
% \endaligned
% $$
% Nous utilisons alors $\Log(1+x)\ge x-x^2/2$ si $x\ge0$ et
% $\Log(1+x)\ge x-x^2$ si $x\ge-\tfrac12$. Il vient
% $$
% \aligned
% \Log(C_{D'}/C_D)
% \ge
% &\sum_{p\le D}
% \frac{\sum_{D<p^{\nu}\le D'}g(p^{\nu})}{1+\sum_{p^{\nu}\le D}g(p^{\nu})}
% -\frac12
% \sum_{p\le D}\left(
% \frac{\sum_{D<p^{\nu}\le D'}g(p^{\nu})}{1+\sum_{p^{\nu}\le D}g(p^{\nu})}
% \right)^2
% \\&+
% \sum_{D<p\le D'}
% \sum_{p^{\nu}\le D'}g(p^{\nu})
% -\frac12
% \sum_{D<p\le D'}
% \left(\sum_{p^{\nu}\le D'}g(p^{\nu})\right)^2
% \\&-\kappa
% \sum_{D<p\le D'}\frac1p-\kappa\sum_{D<p\le D'}\frac1{p^2}
% \\\ge
% &\sum_{p\le D}
% \frac{\sum_{D<p^{\nu}\le D'}g(p^{\nu})}{1+\sum_{p^{\nu}\le D}g(p^{\nu})}
% -\frac12
% \sum_{p\le D}\left(\sum_{D<p^{\nu}\le D'}g(p^{\nu})\right)^2
% \\&+
% \sum_{D<p\le D'}
% \sum_{p^{\nu}\le D'}g(p^{\nu})
% -\frac12
% \sum_{D<p\le D'}
% \left(\sum_{p^{\nu}\le D'}g(p^{\nu})\right)^2
% \\&-\kappa
% \sum_{D<p\le D'}\frac1p-\kappa\sum_{D<p\le D'}\frac1{p^2}.
% \endaligned
% $$
% Nous utilisons alors $1/(1+x)\ge 1-x$ pour obtenir
% $$
% \aligned
% \Log(C_{D'}/C_D)
% \ge&
% \sum_{p\le D}
% \sum_{D<p^{\nu}\le D'}g(p^{\nu})
% -\sum_{p\le D}
% \sum_{D<p^{\nu}\le D'}g(p^{\nu})
% \sum_{p^{k}\le D}g(p^{k})
% -\frac12
% \sum_{p\le D}\left(\sum_{D<p^{\nu}\le D'}g(p^{\nu})\right)^2
% \\&+
% \sum_{D<p\le D'}
% \sum_{p^{\nu}\le D'}g(p^{\nu})
% -\frac12
% \sum_{D<p\le D'}
% \left(\sum_{p^{\nu}\le D'}g(p^{\nu})\right)^2
% \\&-\kappa
% \sum_{D<p\le D'}\frac1p-\kappa\sum_{D<p\le D'}\frac1{p^2}.
% \endaligned
% $$
% Nous r\'earrangeons les termes comme suit~:
% $$
% \aligned
% &-\sum_{p\le D}
% \sum_{D<p^{\nu}\le D'}
% \sum_{p^{k}\le D}
% -\frac12
% \sum_{p\le D}\sum_{D<p^{\nu}\le D'}\sum_{D<p^{k}\le D'}
% -\frac12
% \sum_{D<p\le D'}
% \sum_{p^{\nu}\le D'}\sum_{p^{k}\le D'}
% \\&=
% -\frac12
% \sum_{p\le D'}
% \sum\Sb \nu,k\ge1\\ p^\nu,p^k\le D'\\ p^{\nu}\text{\ ou\ }p^k\ge
% D\endSb
% w_p(\nu,k)
% \endaligned
% $$
% o\`u $w_p(\nu,k)$ vaut
% $$
% w_p(\nu,k)=
% \cases
% 2\quad&\text{si\ }p\le D\quad p^\nu>D\ge p^k,\\
% 1\quad&\text{si\ }p\le D\quad p^\nu,p^k>D,\\
% 1\quad&\text{si\ }p> D.
% \endcases
% $$
% \`A l'aide de nos hypoth\`eses, nous montrons que
% $$
% -
% \sum_{p\le D'}
% \sum\Sb \nu,k\ge1\\ p^\nu,p^k\le D'\\ p^{\nu}\text{\ ou\ }p^k\ge
% D\endSb g(p^{\nu})g(p^k)
% \ge-\frac{2A}{\Log D}
% $$
% et donc
% $$
% \Log(C_{D'}/C_D)
% \ge
% \sum_{p\le D'}
% \sum_{D<p^{\nu}\le D'}g(p^{\nu})
% -\kappa
% \sum_{D<p\le D'}\frac1p-\kappa\sum_{D<p\le D'}\frac1{p^2}
% -\frac{2A}{\Log D}.
% $$
% Supposons \`a pr\'esent que $D$ est un entier (ceci ne restreint en
% rien la g\'en\'eralit\'e de l'argument puisque $\kappa\ge0$), ce qui
% nous garantit la borne
% $$
% \sum_{D<p\le D'}\frac1{p^2}\le\frac1{D}.
% $$
% Par ailleurs une int\'egration par parties sans le moindre probl\`eme
% donne
% $$
% \sum\Sb p\ge2,\nu\ge1\\ D< p^\nu\le D'\endSb g\big(p^\nu\big)
% -\kappa\Log\frac{\Log D'}{\Log D}
% =\Cal O^*\bigg(\frac L{\Log D}\bigg),
% $$
% ce qui nous laisse avec
% $$
% \Log(C_{D'}/C_D)
% \ge
% -\frac{2A+L}{\Log D}-\frac\kappa{D}
% +\kappa\bigg(\Log\frac{\Log D'}{\Log D}-\sum_{D<p\le D'}\frac1p\bigg).
% $$
% Rappelons deux r\'esultats de Rosser \& Schoenfeld (que le lecteur
% trouvera dans l'article cit\'e ci-apr\`es)~:
% $$
% \left\{\aligned
% &\prod_{p\le P}\bigg(1-\frac1p\bigg)^{-1}= e^\gamma\Log P
% \bigg(1+\Cal O^*\big(\frac{1}{2\Log^2 P}\big)\bigg)^{\pm}\qquad(P\ge286)
% \\&
% \sum_{p\le P}\frac1p=\Log\Log P-\gamma-\sum_{p\ge2}\frac{\Log
% p}{p^2-p}
% +\Cal O^*\big(\frac{1}{2\Log^2 P}\big)\qquad(P\ge286)
% \endaligned\right.
% $$
% (o\`u l'on choisit le signe $\pm$)
% ce qui nous donne, en faisant tendre $D'$ vers l'infini
% $$
% \aligned
% G(D)
% &\le \Gamma(\kappa+1)C_D (\Log D)^{\kappa}\, e^{\gamma\kappa}
% \bigg(1+\frac{1}{2\Log^2 D}\bigg)^{\kappa}
% \\&\le
% C(\Log D)^{\kappa}\,\Gamma(\kappa+1)
% \exp\left(
% \frac{2A+L}{\Log D}+\frac{\kappa}{D}+\frac{\kappa}{\Log^2D}
% +\gamma\kappa
% \right)
% \bigg(1+\frac{1}{2\Log^2 D}\bigg)^{\kappa}
% \endaligned
% $$
% pour $D\ge286$. Il nous faut alors montrer que
% $$
% \Gamma(\kappa+1)\exp\left(
% \frac{2A+L}{\Log D}+\frac{\kappa}{D}+\frac{\kappa}{\Log^2D}
% +\gamma\kappa
% \right)
% \bigg(1+\frac{1}{2\Log^2 D}\bigg)^{\kappa}
% \le
% \bigg(1+\frac{B''}{\Log D}\bigg).
% $$
% Pour cela il suffit de
% $$
% (\Log D)\exp\left(
% \frac{2A+L}{\Log D}+\frac{\kappa}{D}+\frac{3\kappa}{2\Log^2D}
% \right)
% \le
% B''e^{-\gamma\kappa}/\Gamma(\kappa+1)
% $$
% soit encore
% $$
% (\Log D)\exp\left(
% \frac{2(A+L)}{\Log D}\right)
% \le \le
% B''e^{-0.7\kappa}/\Gamma(\kappa+1)
% $$
% avec $D\ge300$. Maintenant
% $$
% (\Log D)\exp\left(
% \frac{2(A+L)}{\Log D}\right)\ge 2(L+A)
% $$
% et
% $$
% e^{0.7\kappa}\Gamma(\kappa+1)\ge 1+2(\kappa+1)e^{\kappa+1}.
% $$
\fin


\section{III. Un exemple.}

Consid\'erons la fonction multiplicative $h$ d\'efinie par
$$
\left\{
\aligned
h(p^k)=1\quad\text{si}\quad k\ge 1\quad&\text{et}\quad p\equiv1,2[4],\\
h(p^{2k})=1\quad\text{si}\quad k\ge 1\quad&\text{et}\quad p\equiv3[4],\\
h(p^{2k+1})=1\quad\text{si}\quad k\ge 0\quad&\text{et}\quad p\equiv3[4],\\
\endaligned
\right.
$$
qui est en fait la fonction caract\'eristique des sommes de deux
carr\'es. Avec $g(d)=h(d)/d$, nous constatons que les hypoth\`eses du
th\'eor\`eme~2 sont v\'erifi\'ees pour $\kappa=\tfrac12$ ce qui nous
donne
$$
\sum_{d\le D}\frac{h(d)}{d}=C\,(\Log D)^{1/2}\big(1+\Cal O(1/\Log D)\big)
$$
pour une certaine constante $C>0$.


\Refs

\ref
\paper Mean value theorems for a class of arithmetic functions
\by H. Halberstam \& H. E. Richert
\yr 1971
\jour Acta Arith.
\vol 43
\pages 243--256
\endref

\ref
\paper Sieves methods
\by H. Halberstam \& H. E. Richert
\yr 1974
\jour Academic Press (London)
\pages 364pp
\endref

\ref
\paper On a result of R. R. Hall. 
\by H. Halberstam \& H. E. Richert
\yr 1979
\jour J. Number Theory
\vol 11
\pages 76--89
\endref

\ref
\book Introduction \`a la th\'eorie analytique et probabiliste des
nombres.
2\`eme \'ed.
\by G. Tenenbaum
\yr 1995
\jour Cours Sp\'ecialis\'es. 1. Paris: Soci\'et\'e Math\'ematique de France.
\pages 457pp
\endref

\endRefs

\enddocument

%%% Local Variables: 
%%% mode: plain-tex
%%% TeX-master: t
%%% End: 
