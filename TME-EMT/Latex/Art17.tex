\chapter{  Tools on Mellin transforms}

Corresponding html file: \texttt{../Articles/Art17.html}










\section{Explicit truncated Perron formula}



Here is Theorem 7.1 of
\cite{Ramare*07a}.

\par 
\begin{thm}{Theorem (2007)}

Let $F(z)=\sum_{n}a_n/n^z$ be a Dirichlet series that converges absolutely
  for $\Re z>\kappa_a$, and let $\kappa>0$ be strictly larger than 
  $\kappa_a$. For $x\ge1$ and $T\ge1$, we have
$$
    \sum_{n\le x}a_n
    =\frac1{2i\pi}\int_{\kappa-iT}^{\kappa+iT}F(z)\frac{x^zdz}z
    +\mathcal{O}^*\left(
      \int_{1/T}^{\infty}
      \sum_{|\log(x/n)|\le u}\frac{|a_n|}{n^\kappa}
      \frac{2x^\kappa du}{T u^2}
    \right).
$$
\end{thm}

See
\cite{Ramare*14-6}
for different versions.




\section{L${}^2$-means}


We start with a majorant principle taken for instance from
\cite{Montgomery*94},
chapter 7, Theorem 3.
\par 
\begin{thm}{Theorem}

  Let $\lambda_1,\cdots,\lambda_N$ be $N$ real numbers, and suppose
  that $|a_n|\le A_n$ for all $n$. Then
  $$
  \int_{-T}^T\Bigl|\sum_{1\le n\le N}a_n e(\lambda_n t)\Bigr|^2dt
  \le 3
  \int_{-T}^T\Bigl|\sum_{1\le n\le N}A_n e(\lambda_n t)\Bigr|^2dt
  $$.
\end{thm}

The constant 3 has furthermore been shown to be optimal in
\cite{Logan*88}
where the reader will find an intensive discussion on this
question. The next lower estimate is also proved there:
\par 
\begin{thm}{Theorem}

  Let $\lambda_1,\cdots,\lambda_N$ be $N$ be real numbers, and suppose
  that $a_n\ge 0$ for all $n$. Then
  $$
  \int_{-T}^T\Bigl|\sum_{1\le n\le N}a_n e(\lambda_n t)\Bigr|^2dt
  \ge
  T \sum_{n\le N}a_n^2.
  $$.
\end{thm}



We follow the idea of Corollary 3 of
\cite{Montgomery-Vaughan*74}
but rely on
\cite{Preissmann*84} to get the following.
\par 
\begin{thm}{Theorem (2013)}

  Let $(a_n)_{n\ge1}$ be a series of complex numbers that are such that
  $\sum_n n|a_n|^2 < \infty$ and $\sum_n |a_n| < \infty$. We have, for $T\ge0$,
  \begin{equation*}
    \int_0^T\Bigl|
    \sum_{n\ge1} a_{n}n^{it}
    \Bigr|^{2}dt = 
    \sum_{n\le N}|a_n|^2 \bigl(T+\mathcal{O}^*(2\pi c_0(n+1))\bigr),
  \end{equation*}
  where $c_0=\sqrt{1+\frac23\sqrt{\frac{6}{5}}}$. Moreover, when $a_n$ is
  real-valued, the constant $2\pi c_0$ may be reduced to $\pi c_0$.
\end{thm}

This is Lemma 6.2 from \cite{Ramare*13d}.

\par 
Corollary 6.3 and 6.4 of
\cite{Ramare*13d}
contain explicit versions of a Theorem of
\cite{Gallagher*70}

\par 
\begin{thm}{Theorem (2013)}

  Let $(a_n)_{n\ge1}$ be a series of complex numbers that are such that
  $\sum_n n|a_n|^2 < \infty$ and $\sum_n |a_n| < \infty$. We have, for $T\ge0$,
  $$
  \sum_{q\le Q}\frac{q}{\varphi(q)}
  \sum_{\substack{\chi\mod q,\\ \text{$\chi$ primitive}}}
  \int_{-T}^T
    \biggl|\sum_{n}a_n \chi(n)n^{it}\biggr|^2dt
    \le
    7
    \sum_{n}|a_n|^2( n+ Q^2\max(T, 3) ).
  $$
\end{thm}


\par 
\begin{thm}{Theorem (2013)}

  Let $(a_n)_{n\ge1}$ be a series of complex numbers that are such that
  $\sum_n n|a_n|^2 < \infty$ and $\sum_n |a_n| < \infty$. We have, for $T\ge0$,
  $$
  \sum_{q\le Q}\frac{q}{\varphi(q)}
  \sum_{\substack{\chi\mod q,\\ \text{$\chi$ primitive}}}
  \int_{-T}^T
    \biggl|\sum_{n}a_n \chi(n)n^{it}\biggr|^2dt
    \le
    \sum_{n}|a_n|^2( 43n+ \tfrac{33}{8} Q^2\max(T, 70) ).
  $$
\end{thm}




 
 








  
\begin{flushright}\small\sl{}   Last updated on July 14th, 2013, by Olivier Ramar\'e
 \end{flushright}















