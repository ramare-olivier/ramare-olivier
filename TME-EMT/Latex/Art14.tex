\chapter{  Sieve bounds}

Corresponding html file: \texttt{../Articles/Art14.html}










 
 

\par 
\section{Some upper bounds}


Theorem 2 of \cite{Montgomery-Vaughan*73} contains the
following explicit version of the Brun-Tichmarsh Theorem.
\par 
\begin{thm}{Theorem (1973)}

Let $x$ and $y$ be positive real numbers, and let $k$ and $\ell$ be relatively
prime positive integers. Then 
$
\pi(x+y;k,\ell)-\pi(x;k,\ell)
<  \frac{2y}{\phi(k)\log (y/k)}
$ provided only that $y>k$.
\end{thm}

Here as usual, we have used the notation
$$
\pi(z;k,\ell)=\sum_{\substack{p\le z,\\ p\equiv \ell [k]}}1,
$$
i.e. the number of primes up to $z$ that are coprime to $\ell$ modulo $k$.
See
\cite{Buethe*14}
for a generic weighted version of this inequality.

\par 
Here is a bound concerning a sieve of dimension 2 proved by
\cite{Siebert*76}.
\par 
\begin{thm}{Theorem (1976)}

Let $a$ and $b$ be coprime integers with $2|ab$. Then we have, for $x>1$,
$$
\sum_{\substack{p\le x,\\ \text{$ap+b$ prime}}}1
\le 16 \omega\frac{x}{(\log x)^2}\prod_{\substack{p|ab,\\ p >
2}}\frac{p-1}{p-2}
\qquad \omega=\prod_{p > 2}(1-(p-1)^{-2}).
$$
\end{thm}



\section{Combinatorial sieve estimates}


The combinatorial sieve is known to be difficult from an explicit
viewpoint. For the linear sieve, the reader may look at Chapter 9,
Theorem 9.7 and 9.8 from
\cite{Nathanson*96-2}.







  
\begin{flushright}\small\sl{}   Last updated on July 18th, 2013, by Olivier Ramar\'e
 \end{flushright}














